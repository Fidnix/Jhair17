%\renewcommand{\baselinestretch}{1.5}
\onehalfspacing
\fancypagestyle{plain}{
\fancyhf{}
\rhead{\thepage}
\renewcommand{\headrulewidth}{0pt}
\renewcommand{\footrulewidth}{0pt}
}

\chapter{Generalidades}

\section{Problemática}

\subsection{Árbol de problemas}

\begin{table}[H]
    \centering
    % Obs: Transformación de los problemas
    % 2 & 3 se juntan
    % 1 -> 2
    \begin{tabularx}{\textwidth}{|c|X|X|X|}
    \hline
    % Consecuencias
    Consecuencias & Dificultad de los alumnos para continuar sus ciclos escolares debido a la inaccesibilidad a material educativo útiles para su comprensión de los temas que requieren & Genera que muchos estudiantes continúen repitiendo ciclos escolares o, incluso, que deserten de su educación escolar & Dificultad para aprovechar las tecnologías educativas disponibles actualmente\\
    \hline
    % Central
    Problema central & \multicolumn{3}{|>{\hsize=3\hsize}X|}{Los estudiantes en comunidades rurales suelen presentar dificultades en el aprendizaje de ciencias debido a la falta de materiales didácticos contextualizados y metodologías poco interactivas} \\
    \hline
    % Causas
    Causas & Las escuelas rurales carecen de materiales pedagógicos para la enseñanza en las escuelas rurales según \cite[p.~136]{montes2023participacion} & Existen motivacionales en los estudiantes según \cite[p.~161]{Torres_Gonzalez_Acevedo_Correa_Gallo_Garcia_2016}. A su vez, la descontextualización curricular en la pedagogía y evaluaciones que se dan a escuelas en zonas rurales incitan a la deserción y bajo rendimiento escolar, según \cite[p.~23] {elaner_antonio_arrieta_vega_2024_13517936} & Problemas de accesibilidad a tecnologías educativas como al programa "{Aprendo en casa}", según \cite[p.~98-99]{renato_educacion} por problemas de inherentes a la infraestructura de telecomunicaciones local y el apoyo efímero estatal con respecto a estos temas \\ % En problema 2 debe ir un citado de segunda mano
    \hline
    \end{tabularx}

    \caption{Árbol de problemas}
    \label{tab:arbol_problemas}
\end{table}

\subsection{Descripción}

\newcounter{siglo}
\setcounter{siglo}{20}

Según \cite[p.~140-143]{elaner_antonio_arrieta_vega_2024_13517936} actualmente la enseñanza de la ciencias y las tecnologías es muy importante ya que dota a los estudiantes de capacidad en pensamiento crítico y creativo, sobre todo cuando la última mitad del siglo \Roman{siglo}, en que la ciencia y la tecnología cambiaron el modo de ver del mundo y están más presentes que nunca.

No obstante, aun se existen problemas respecto a la educación rural; prueba de ello, se puede dar en las evaluaciones PISA. En primer lugar, se debe mencionar que en dichas evaluaciones se categoriza el rendimiento de los estudiantes en ciencias entre un nivel de $1$b a $6$, dependiendo de la materia correspondiente, donde se observa que en los resultados en evaluación de ciencias en las escuelas peruanas, el $52.9\%$ los estudiantes de zonas urbanas se ubicaron en el nivel 2 a más; mientras, que en las escuelas rurales, tan solo el $21.8\%$ de sus estudiantes llegan al nivel 2 a más, \cite[p.~32]{umcPISA2022presentation}. Ello refleja el bajo rendimiento que pueden estar sucediendo por diversas circunstancias en las escuelas rurales.

Uno de los factores que generan y mantienen el bajo desempeño escolar de varios estudiantes es la carencia de materiales y pedagogías en varias escuelas rurales, según \cite[p.~136]{montes2023participacion}; estas carencias son a nivel estructural, accesibilidad o, lo concerniente al presente proyecto, recursos educativos. Este factor, perjudica a los estudiantes; debido a que no tienen acceso a una educación regular y, por ende, les dificulta la comprensión de temas.

Por otro lado, un problema que también está presente en la realidad de las comunidades rurales es la deserción de los ciclos escolares. Esto puede tener muchos factores como la situación socioeconómica, condiciones laborales o por el contexto familiar; incluso también existen problemas motivacionales, personales y psico-afectivos que impulsan los retrocesos educativos de varios estudiantes, como lo menciona \cite{Torres_Gonzalez_Acevedo_Correa_Gallo_Garcia_2016}.

Otro problema son las clases poco didácticas y además las pruebas estandarizadas que se realizan tiene poco ajuste a los contextos culturales de los estudiantes, por lo que también se vuelve en una dificultad para ellos en el tema de la educación, según \cite[p.~23]{elaner_antonio_arrieta_vega_2024_13517936}.

Además, como menciona \cite{renato_educacion}, actualmente existe apoyo para la enseñanza de educación rural, una de ellas es el programa estatal "Aprendo en casa", cuya premisa es proveer educación remota originada por el estado de emergencia durante la pandemia. No obstante, no resultó ser una solución inmediata, debido a que las zonas rurales, al carecer de redes de comunicaciones, no permite el acceso a dichos programas a los estudiantes. 

\subsection{Problema seleccionado}

Por el árbol de problemas elaborado anteriormente, se plantea como problema central que los estudiantes en comunidades rurales suelen presentar dificultades en el aprendizaje de ciencias debido a la falta de materiales didácticos contextualizados y metodologías poco interactivas.

\section{Objetivos}

\subsection{Objetivo general}

El propósito principal del presente trabajo de tesis es desarrollar un videojuego educativo enfocado a enseñar ciencias en comunidades rurales, que incorpore soporte bilingüe y presente elementos culturales regionales.

\subsection{Objetivos específicos}

\begin{enumerate}[label=O\arabic*.,itemsep=5pt]
    % Obs: Identificacion de topicos
    % Identificcacion de mecanicas de aprendizaje
    % Catalogo de contenidos en Ciencias y tecnologías. Tmb puede ser visto como resultado
    \item Identificación de temas principales para la enseñanza de Ciencias y Tecnología
    \item Identificar las principales competencias requeridas para la enseñanza de Ciencias y tecnologías
    \item Identificar mecánicas de aprendizaje que permitan a los estudiantes engancharse y automotivarse con respecto a sus estudios dentro del videojuego
    \item Desarrollar un videojuego accesible que incorpore mecánicas de aprendizaje para enseñar Ciencias y tecnologías. Y que pueda ser utilizado con las tecnologías disponibles en comunidades rurales.
    % Obs: Considerar dentro del diseño las mecanicas / Identificacion de catalogos
    % Obs: Desarrollar -> Diseñar. Puede ser desarollo o construccion
    \item Incorporar soporte bilingüe y elementos culturales en el desarrollo del videojuego para promover una enseñanza de las ciencias que contextualice la realidad de los estudiantes en comunidades rurales.
    
    % TODO: Revisar bien el alcance de este objetivo. Dentro de los catalogos definidos en resultados anteriores
    \item Lograr que los estudiantes de comunidades rurales cumplan las competencias en la materia de Ciencias y tecnologías establecidas en el Currículo Nacional del Perú mediante el uso de un videojuego educativo interactivo y contextualizado.
\end{enumerate}

\subsection{Resultados esperados}

%%
% \begin{enumerate}[label=O\arabic*,itemsep=5pt]
%    \item Diseñar un videojuego educativo que integre contenidos que aporten al desarrollo de las competencias de Ciencias y Tecnologías.
%    \begin{enumerate}[label=R\arabic{enumi}.\arabic*.]
%        \item 
%        \item Arquitectura del videojuego educativo
%        \item Videojuego desarrollado
%        \item Resultados de competencias de aprendizaje logrados mediante el videojuego
%    \end{enumerate}
%    \item Implementar en el videojuego mecánicas de aprendizaje con el fin de que los estudiantes fortalezcan sus motivaciones y sean más eficaces con el aprendizaje de las ciencias y tecnologías.
%    \item Desarrollar un videojuego educativo con estrategias pedagógicas activas y adaptadas a los contextos rurales, utilizando recursos visuales, narrativas y mecánicas lúdicas que faciliten el aprendizaje significativo en entornos con escasos materiales educativos mediante el soporte bilingüe o la integración de elementos culturales.
%\end{enumerate}

\begin{enumerate}[label=O\arabic*.,itemsep=5pt]
    % Obs: Identificacion de topicos
    % Identificcacion de mecanicas de aprendizaje
    % Catalogo de contenidos en Ciencias y tecnologías. Tmb puede ser visto como resultado
    \item Identificación de temas principales para la enseñanza de Ciencias y Tecnología
    
    \begin{enumerate}[label=R\arabic{enumi}.\arabic*.]
        \item Catalogo de temas principales para la enseñanza de Ciencias y Tecnología.
        \item Informe comparativo entre los temas seleccionados y los contenidos del Currículo Nacional del Perú.
        \item Tabla de priorización de temas según pertinencia en comunidades rurales.
    \end{enumerate}

    \item Identificar las principales competencias requeridas para la enseñanza de Ciencias y tecnologías.
    
    \begin{enumerate}[label=R\arabic{enumi}.\arabic*.]
        \item Catalogo de competencias requeridas para la evaluación de rendimiento de aprendizaje de ciencias y tecnologías que correspondan al plan de estudios nacional del Perú en Ciencias y Tecnología.
        \item Informe de correspondencia entre competencias y habilidades desarrollables mediante videojuegos educativos. % Revisar bien (sera tabla)
        \item Documento de indicadores observables para medir el logro de competencias con videojuegos.
    \end{enumerate}

    \item Identificar mecánicas de aprendizaje que permitan a los estudiantes engancharse y automotivarse con respecto a sus estudios dentro del videojuego
    
     \begin{enumerate}[label=R\arabic{enumi}.\arabic*.]
        \item Catalogo de mecánicas de aprendizaje aplicables a videojuegos.
        \item Documento de análisis de motivación intrínseca y extrínseca asociada a cada mecánica.
        % R3.3. Quizás pueda ayudar para la medicicón de resultados finales
        \item Prototipos de mini-juegos o dinámicas que ejemplifiquen la aplicación de mecánicas seleccionadas.
    \end{enumerate}

    \item Desarrollar un videojuego accesible que incorpore mecánicas de aprendizaje para enseñar Ciencias y tecnologías. Y que pueda ser utilizado con las tecnologías disponibles en comunidades rurales.
    
    \begin{enumerate}[label=R\arabic{enumi}.\arabic*.]
        \item Documento de diseño de videojuego para definir y elaborar los aspectos conceptuales y de alto nivel del videojuego educativo
        \item Documento técnico de requisitos mínimos para ejecutar el videojuego en hardware local.
        \item Prototipo funcional del videojuego con al menos un módulo jugable.
        \item Informe de pruebas de compatibilidad en distintos dispositivos (PC, móvil, tablet).
    \end{enumerate}
    % Obs: Considerar dentro del diseño las mecanicas / Identificacion de catalogos
    % Obs: Desarrollar -> Diseñar. Puede ser desarollo o construccion
    \item Incorporar soporte bilingüe y elementos culturales en el desarrollo del videojuego para promover una enseñanza de las ciencias que contextualice la realidad de los estudiantes en comunidades rurales.
    
    \begin{enumerate}[label=R\arabic{enumi}.\arabic*.]
        
        \item Informe de validación cultural y lingüística en colaboración con miembros de la comunidad.
        % Obs: No enfocarse tanto en el tema técnico
        % El objetivo es el aprendizaje y el diseño del videojuego
        %\item Documento que defina y justifique la arquitectura y el diseño de software del videojuego
        % Obs: Game design (investigarlo)
    \end{enumerate}

    \item Lograr que los estudiantes de comunidades rurales cumplan las competencias en la materia de Ciencias y tecnologías establecidas en el Currículo Nacional del Perú mediante el uso de un videojuego educativo interactivo y contextualizado.
    
    \begin{enumerate}[label=R\arabic{enumi}.\arabic*.]
        % OBS: alcance enorme. 
        \item Resultados de cumplimiento de competencias educativas con respecto al plan de estudios nacional del Perú en Ciencias y Tecnología.
        \item Retroalimentación cualitativa de estudiantes y docentes sobre la experiencia de aprendizaje con el videojuego.
    \end{enumerate}
\end{enumerate}

% \begin{enumerate}[label=R\arabic*.,itemsep=5pt]
    
%     \item Catalogo de competencias requeridas para la evaluación de rendimiento de aprendizaje de ciencias y tecnologías que correspondan al plan de estudios nacional del Perú en Ciencias y Tecnología
%     \item Catalogo de mecánicas de aprendizaje aplicables a videojuegos
%     \item Documento de diseño de videojuego para definir y elaborara los aspectos conceptuales y de alto nivel del videojuego educativo
%     \item Videojuego educativo para el aprendizaje de ciencias que integre soporte bilingüe e integre elementos culturales
%     % Obs: No enfocarse tanto en el tema técnico
%     % El objetivo es el aprendizaje y el diseño del videojuego
%     %\item Documento que defina y justifique la arquitectura y el diseño de software del videojuego
%     % Obs: Game design (investigarlo)

%     \item Resultados de cumplimiento de competencias educativas con respecto al plan de estudios nacional del Perú en Ciencias y Tecnología
% \end{enumerate}


\subsection{Mapeo de objetivos, resultados y verificación}

% TODO: Otros indicadores: Catalogo cuenta con tantos items...
% El numero de items en catalogo deberia ser reazonable
% Con el curriculum nacional por ejemplo, con sus competencias
\begin{table}[h]
\centering
\renewcommand{\arraystretch}{1.4}
\begin{tabularx}{\textwidth}{|c|X|X|X|}
\hline
\textbf{Objetivo} & \textbf{Resultado esperado} & \textbf{Medio de verificación} & \textbf{Indicador verificable} \\
\hline

\multirow{3}{*}{O1} 
& R1.1. Catálogo de temas principales para la enseñanza de Ciencias y Tecnología. 
& Documento entregable. 
& Aprobación al 100\% por especialistas en Educación. \\ \cline{2-4}

& R1.2. Informe comparativo entre los temas y el Currículo Nacional. 
& Informe escrito. 
& Validación $\geq$ 80\% de especialistas en currículo educativo. \\ \cline{2-4}

& R1.3. Tabla de priorización de temas según pertinencia en comunidades rurales. 
& Tabla validada por docentes locales. 
& Aprobación por al menos 2 expertos del área. \\ 
\hline

\multirow{3}{*}{O2} 
& R2.1. Catálogo de competencias requeridas. 
& Documento técnico. 
& Documento validado por especialistas en educación. \\ \cline{2-4}

& R2.2. Informe de correspondencia entre competencias y habilidades mediante videojuegos. 
& Informe técnico. 
& Validación $\geq$ 80\% de pertinencia por expertos. \\ \cline{2-4}

& R2.3. Documento de indicadores observables para medir logro de competencias. 
& Documento con rúbricas. 
& Claridad y pertinencia $\geq$ 80\% según docentes. \\ 
\hline

\multirow{3}{*}{O3} 
& R3.1. Catálogo de mecánicas de aprendizaje aplicables a videojuegos. 
& Documento entregable. 
& Validación de al menos 2 expertos en gamificación. \\ \cline{2-4}

& R3.2. Documento de análisis de motivación intrínseca y extrínseca. 
& Informe escrito. 
& Aprobación $\geq$ 80\% de especialistas en educación. \\ \cline{2-4}

& R3.3. Prototipos de mini-juegos aplicando mecánicas seleccionadas. 
& Prototipos digitales o maquetas. 
& 100\% de prototipos funcionales evaluados por docentes y estudiantes. \\ 
\hline

\multirow{4}{*}{O4} 
& R4.1. Documento de diseño conceptual y de alto nivel del videojuego educativo. 
& Documento técnico (game design document). 
& Validación por al menos 2 especialistas en desarrollo de videojuegos. \\ \cline{2-4}

& R4.2. Documento de requisitos mínimos para ejecución en hardware local. 
& Documento técnico. 
& Cumplimiento del 100\% de especificaciones. \\ \cline{2-4}

& R4.3. Prototipo funcional del videojuego con al menos un módulo jugable. 
& Ejecutable del prototipo. 
& Superar el 100\% de pruebas unitarias e integración. \\ \cline{2-4}

& R4.4. Informe de pruebas de compatibilidad en distintos dispositivos. 
& Reporte técnico. 
& Ejecución exitosa en $\geq$ 80\% de los dispositivos rurales identificados. \\ 
\hline

\multirow{1}{*}{O5} 
& R5.1. Informe de validación cultural y lingüística en colaboración con la comunidad. 
& Documento avalado por actores locales. 
& Validación $\geq$ 90\% de adecuación cultural y lingüística. \\ 
\hline

\multirow{2}{*}{O6} 
& R6.1. Resultados de cumplimiento de competencias educativas respecto al Currículo Nacional. 
& Evaluación pre y post test a estudiantes. 
& Mejora $\geq$ 20\% en competencias en comparación a línea base. \\ \cline{2-4}

& R6.2. Retroalimentación cualitativa de estudiantes y docentes sobre la experiencia de aprendizaje. 
& Encuestas, entrevistas y grupos focales. 
& $\geq$ 80\% de satisfacción en estudiantes y docentes. \\ 
\hline

\end{tabularx}
\caption{Matriz de Objetivos, Resultados, Medios de Verificación e Indicadores}
\end{table}

\section{Métodos y procedimientos}