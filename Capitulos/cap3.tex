\renewcommand{\baselinestretch}{1.5}
\onehalfspacing
\fancypagestyle{plain}{
\fancyhf{}
\rhead{\thepage}
\renewcommand{\headrulewidth}{0pt}
\renewcommand{\footrulewidth}{0pt}
}

\chapter{Estado del arte}

\section{Introducción}

Este capítulo tiene por objetivo identificar estudios previos relacionados al desarrollo de videojuegos educativos; sobre todo, aquellos enfocados en los temas de Ciencia y tecnología. Para ello se realizará una revisión sistemática de literatura para 

\section{Objetivos de revisión}

\begin{itemize}
    \item Conocer los conceptos esenciales sobre el desarrollo de videojuegos
    \item Conocer los conceptos esenciales sobre los videojuegos educativos o serios
    \item Conocer las principales mecánicas que usan los videojuegos educativos
    \item Conocer cómo los videojuegos educativos resultan eficaces en la enseñanza de ciencia
    \item Conocer las necesidades y retos en el aprendizaje de ciencias en escuelas rurales
    \item Conocer las métricas que permitan comprobar la efectividad del aprendizaje
    \item Conocer qué metodologías se utilizan en el desarrollo de videojuegos en general
    \item Identificar la integración de elementos culturales dentro de los videojuegos
    \item Identificar soluciones que sean videojuegos educativos orientados a enseñar ciencias en la actualidad
\end{itemize}

\section{Preguntas de revisión}

La elaboración de las preguntas de revisión se realizó a partir de la metodología PICOC; en particular, mediante los criterios de población, intervención, comparación, resultados y contexto.

\begin{table}[H]
    \centering
    \begin{tabularx}{\textwidth}{|c|c|X|}
        \hline
        Criterio & Explicación & Descripción \\ \hline
        Población & ¿Quién? & Estudiantes de comunidades rurales \\ \hline
        Intervención & ¿Qué o cómo? & Uso de un videojuego educativo diseñado con soporte bilingüe y con integración de elementos culturales propios de una comunidad \\ \hline
        Comparación & ¿Comparado con qué? & Métodos tradicionales de enseñanza (clases magistrales, materiales impresos, dinámicas de aula sin uso de videojuegos) \\ \hline
        Resultados & ¿Qué queremos lograr? & Mejora en la comprensión de conceptos de ciencia, incremento en la motivación y el compromiso con el aprendizaje \\ \hline
        Contexto & Educación en zonas rurales & Educación en comunidades rurales con condiciones de acceso limitado a tecnología y necesidad de reforzar aprendizajes científicos \\ \hline
    \end{tabularx}
    \caption{Tabla PICOC}
    \label{tab:placeholder}
\end{table}

A partir de la tabla PICOC, se formulan las siguientes preguntas:

\begin{enumerate}
    \item ¿Qué elementos y metodologías existen sobre el desarrollo de videojuegos educativos?
    \item ¿Qué mecánicas poseen los videojuegos educativos para poder enseñar temas educativos, especialmente los temas de ciencia y tecnología?
    \item ¿Qué evidencia y metodologías de resultados existen sobre la eficacia de los videojuegos educativos en la enseñanza de ciencias en contextos rurales o con recursos limitados?
    \item ¿Cómo los videojuegos educativos pueden integrar elementos culturales en su diseño?
    \item ¿Qué soluciones actuales se desarrollaron para la enseñanza de ciencias mediante el uso de videojuegos educativos?
\end{enumerate}

\section{Estrategia de búsqueda}

\subsection{Motores de búsqueda a usar}

Se usaron los siguientes motores de búsqueda para la búsqueda de los documentos relevantes:

\begin{itemize}
    \item Scopus
    \item IEEE Xplore
    \item Web of Science
    \item Dialnet % verify
\end{itemize}

\subsection{Cadenas de búsqueda a usar}

Con el propósito de encontrar documentos pertinentes para la resolución de las preguntas de revisión se utilizarán las cadenas de búsqueda; pues estas permiten encontrar información mediante cadenas lógicas.

\begin{enumerate}[label=\textbf{Cadena \arabic*},leftmargin=6em]
    \item enfocada en identificar fuentes que permitan conocer los desafìos actuales de la educación rural: \textit{(``educacion inclusiva'' OR ``brecha educativa'') AND (``grupo minoritario'' OR ``zona rural'' OR ``rural'') AND (``Peru'')}
    \item enfocada en identificar fuentes que se refieran al desarrollo o diseño de videojuegos educativos orientados a la educación STEM: \textit{(develop* OR design*) AND (educat* OR serious*) AND (videogame* OR ``video game*'' OR ``digital game'') AND (school OR institut*) AND (instruct* OR teach* OR educat* OR learn*) AND (science OR stem)}
    %\begin{itemize}
    %    \item Scopus: TITLE-ABS-KEY ( ( develop* OR design* ) AND ( educat* OR serious* ) AND ( videogame* OR "video game*" OR "digital game" ) AND ( school OR institut* ) AND ( instruct* OR teach* OR educat* OR learn* ) AND ( science OR stem ) ) AND PUBYEAR > 2017 AND PUBYEAR < 2027 AND ( LIMIT-TO ( SUBJAREA , "SOCI" ) OR LIMIT-TO ( SUBJAREA , "COMP" ) OR LIMIT-TO ( SUBJAREA , "ENGI" ) OR LIMIT-TO ( SUBJAREA , "MATH" ) OR LIMIT-TO ( SUBJAREA , "DECI" ) OR LIMIT-TO ( SUBJAREA , "ARTS" ) ) AND ( LIMIT-TO ( LANGUAGE , "English" ) OR LIMIT-TO ( LANGUAGE , "Spanish" ) )
    %\end{itemize}
    \item enfocada en identificar fuentes que integren elementos culturales varios dentro de los videojuegos: \textit{(cultur* OR hist*) AND (educat* OR serious*) AND (videogame* OR videogame* OR ``video game*'' OR ``digital game'')}
   % \begin{itemize}
        %\item Scopus: TITLE-ABS-KEY ( ( cultur* OR hist* ) AND ( educat* OR serious* ) AND ( videogame* OR videogame* OR "video game*" OR "digital game" ) ) AND PUBYEAR > 2017 AND PUBYEAR < 2026 AND ( LIMIT-TO ( SUBJAREA , "COMP" ) OR LIMIT-TO ( SUBJAREA , "ENGI" ) ) AND ( LIMIT-TO ( LANGUAGE , "English" ) OR LIMIT-TO ( LANGUAGE , "Spanish" ) ) AND ( LIMIT-TO ( EXACTKEYWORD , "Serious Games" ) OR LIMIT-TO ( EXACTKEYWORD , "Human Computer Interaction" ) OR LIMIT-TO ( EXACTKEYWORD , "Students" ) OR LIMIT-TO ( EXACTKEYWORD , "Video-games" ) OR LIMIT-TO ( EXACTKEYWORD , "Game Design" ) OR LIMIT-TO ( EXACTKEYWORD , "Video Games" ) OR LIMIT-TO ( EXACTKEYWORD , "E-learning" ) OR LIMIT-TO ( EXACTKEYWORD , "Game-based Learning" ) OR LIMIT-TO ( EXACTKEYWORD , "Cultural Heritages" ) OR LIMIT-TO ( EXACTKEYWORD , "Cultural Heritage" ) OR LIMIT-TO ( EXACTKEYWORD , "Learning Systems" ) OR LIMIT-TO ( EXACTKEYWORD , "Digital Game-based Learning" ) OR LIMIT-TO ( EXACTKEYWORD , "Serious Game" ) OR LIMIT-TO ( EXACTKEYWORD , "Game-based" ) )
        %\item IEEE Xplore: (("All Metadata":"develop*" OR "All Metadata":"design*") AND ("All Metadata":"educat*" OR "All Metadata":"serious*") AND (videogame* OR "video game*" OR "digital game") AND ("All Metadata":"school") AND ("All Metadata":"teach*" OR "All Metadata":"learn*") AND ("All Metadata":"science" OR "All Metadata":"stem"))
    %\end{itemize}
    % \item Cadena 3 enfocada en identificar fuentes de desarrollo de videojuegos bilingües o multilingüísticos: \textit{(develop* OR design*) AND (educat* OR serious*) AND (videogame* OR "video game*" OR "digital game") AND (school OR inst*) AND (bilingual OR multilingual)}
    \item enfocado en identificar fuentes que provean información sobre el impacto o relevancia de los videojuegos serios en la educación rural: \textit{(``educational game*'' OR ``serious game*'' OR ``learning game*'' OR gamification) AND ( effectiveness OR impact OR efficacy OR evaluation OR outcome* OR assessment ) AND ( rural OR ``rural education'' OR ``remote area*'' OR ``underserved communit*'' OR ``rural school*'' )}

    %\begin{itemize}
     %   \item Scopus: TITLE-ABS-KEY ( ( "educational game*" OR "serious game*" OR "learning game*" OR gamification ) AND ( effectiveness OR impact OR efficacy OR evaluation OR outcome* OR assessment ) AND ( rural OR "rural education" OR "remote area*" OR "underserved communit*" OR "rural school*" ) ) AND PUBYEAR > 2017 AND PUBYEAR < 2026 AND ( LIMIT-TO ( SUBJAREA , "SOCI" ) ) AND ( LIMIT-TO ( EXACTKEYWORD , "Gamification" ) OR LIMIT-TO ( EXACTKEYWORD , "Serious Games" ) OR LIMIT-TO ( EXACTKEYWORD , "Students" ) OR LIMIT-TO ( EXACTKEYWORD , "Serious Game" ) OR LIMIT-TO ( EXACTKEYWORD , "Game Design" ) OR LIMIT-TO ( EXACTKEYWORD , "Education" ) OR LIMIT-TO ( EXACTKEYWORD , "Engineering Education" ) OR LIMIT-TO ( EXACTKEYWORD , "Game-based Learning" ) OR LIMIT-TO ( EXACTKEYWORD , "Computer Games" ) OR LIMIT-TO ( EXACTKEYWORD , "Educational Game" ) ) AND ( LIMIT-TO ( LANGUAGE , "English" ) )
     %   \item IEEE Xplore: (("All Metadata":"develop*" OR "All Metadata":"design*") AND ("All Metadata":"educat*" OR "All Metadata":"serious*") AND ("educational game*" OR "video game*" OR "digital game") AND ("All Metadata":"school") AND ("All Metadata":"teach*" OR "All Metadata":"learn*") AND ("All Metadata":"science" OR "All Metadata":"stem"))
    %\end{itemize}
\end{enumerate}

\subsection{Documentos encontrados}

\begin{table}[H]
    \centering
    \begin{tabularx}{\textwidth}{|c|c|c|c|c|X|X|}
        \hline
        Motor de búsqueda  & Cadena 1 & Cadena 2 & Cadena 3 & Cadena 4 & Cantidad de documentos encontrados & Cantidad de documentos seleccionados \\ \hline
        Scopus & - &  205 & 317 & 17 & 539 & \\ \hline
        IEEE Xplore & - & 215 & 407 & 5 & 627 & \\ \hline
        Web of Science & - & 215 & 407 & 5 & 627 & \\ \hline
        Dialnet & 100 & - & - & - & - & \\ \hline
        % Web of Science & & & & & \\ \hline
    \end{tabularx}
    \caption{Resultados de búsqueda}
    \label{tab:placeholder}
\end{table}

\subsection{Criterios de inclusión/exclusión}

Para la selección de documentos se establecieron criterios de inclusión que permiten acotar la búsqueda y asegurar la identificación de estudios pertinentes para responder las preguntas planteadas, estos criterios son:

\begin{itemize}
    \item Los documentos deben tener una antigüedad no mayor a 7 años
    \item Se incluirán documentos que tengan por idioma el inglés o el español
    \item Se limitará la búsqueda a documentos que estén relacionados a temas de ciencias sociales en caso de los documentos de la cadena 3, o ciencias de la computación, ingeniería y ciencias naturales en el caso de las otras cadenas
    \item Para el caso de Web of Science, se incluirán los primeros 200 documentos encontrados
\end{itemize}

Por otra parte, se añadieron criterios de exclusión con el fin de excluir los documentos y estudios encontrados que no sean relevantes para la resolución de preguntas pertinentes, estos criterios son:

\begin{itemize}
    \item Documentos cuya fecha de publicación sea anterior a 7 años
    \item Documentos que fueron escritas en idiomas que no sean español o inglés
    \item Documentos sobre desarrollo de videojuegos que tratan temas relacionados a la medicina, exceptuando la psicología y salud mental
    \item Documentos cuyos temas tratan temas de desarrollo de sistemas de inteligencia artificial o algoritmos basados en datos
    \item Documentos cuyos temas tratan sobre sistemas de bajo nivel
\end{itemize}

\section{Formulario de extracción de datos}

\begin{table}[H]
    \centering
    \begin{tabularx}{\textwidth}{|X|X|}
        \hline
        Nombre & Descripción \\ \hline
        ID & Número de identificación del documento dentro del formulario de extracción \\ \hline
        Título & Nombre explícito del documento \\ \hline
        Autores & Nombre de los autores \\ \hline
        Tipo de documento & Tipo de documento como revista, artículo, etc. \\ \hline
        Motor de búsqueda & Motor de búsqueda de proveniencia \\ \hline
        Lenguaje & Lenguaje con el que fue escrito el documento \\ \hline
        Metodología & Metodología de desarrollo \\ \hline
        Objetivo del estudio & Objetivo general del estudio \\ \hline
        Mecánicas de juego & Mecánicas de juego usadas en el desarrollo \\ \hline
        Metodología & Metodología de desarrollo \\ \hline
        Tema del videojuego & En caso sea un videojuego educativo, ¿Qué temas aborda en sí? \\ \hline
        Elementos culturales & Qué elementos están presentes de la cultura que representa, en caso tratara de ello \\ \hline
        Logros de aprendizaje esperados & Qué competencias o aprendizajes se espera que el estudiante aprenda con el medio \\ \hline
        Metodología de resultados & Qué pruebas se tomó en cuenta para conocer la eficacia del medio, en caso se trate de un videojuego educativo \\ \hline
        
    \end{tabularx}
    \caption{Formulario de extracción}
    \label{tab:definicion_formulario_de_extraccion}
\end{table}

\section{Resultados de la revisión}

\begin{enumerate}
    \item ¿Qué elementos y metodologías existen sobre el desarrollo de videojuegos educativos?
    
    Según \cite{}, el proceso de desarrollo de los videojuegos se relaciona al proceso de desarrollo de cualquier otro software, debido a que un videojuego no es más que un software con características particulares. Y como tal, el desarrollo de videojuegos puede seguir metodologías tradicionales como Cascada, Espiral, RUP, entre otras; incluso seguir ciertos elementos presentes en el desarrollo común de software como la reutilización de componentes, pruebas unitarias, integración continua, etc.
    
    Los videojuegos, a pesar de sus bases en el desarrollo de software, tiene ciertas particularidades que lo diferencian del desarrollo de otro tipo de software, como por ejemplo, la presencia de elementos artísticos y creativos, los cuales no son comunes en otro tipo de software. Esto motivó a varios estudios y autores por buscar metodologías específicas para el desarrollo de videojuegos. Además, según \cite{}, las metodologías pueden variar por el tipo de videojuego que se desee desarrollar, ya que no es lo mismo desarrollar un videojuego AAA (videojuego de alta calidad y presupuesto) que un videojuego indie (videojuego desarrollado por un equipo pequeño o incluso una sola persona).

    Una de las metodologías encontradas en la revisión sistemática es GDLC (Game Development Life Cycle) propuesta por \cite{}, la cual está basada en la metodología en cascada pero adaptada a las necesidades del desarrollo de videojuegos. Según \cite{}, esta metodología fue diseñada para cubrir aspectos del desarrollo de videojuegos como el diseño, la programación, el arte y la música. No obstante, según \cite{}, esta metodología no tiene un proceso estándar para el desarrollo de videojuegos, debido a que varios equipos de desarrollo lo adaptan de distintas maneras según sean sus necesidades.

    Según \cite{}, las etapas recomendadas para el desarrollo de videojuegos bajo la metodología GDLC se presentan a continuación y se ilustran en la Figura~\ref{fig:gdlc_stages}.:

    \begin{enumerate}
        \item Inicio: Esta etapa destaca debido a que el desarrollador investiga y define los conceptos principales del videojuego como el contexto y las premisas detrás. Puede incluir la búsqueda de la exploración de las necesidades de los jugadores, según \cite{}.
        \item Pre-producción: En esta etapa se inicia el diseño del videojuego, definiendo las mecánicas de juego, la historia, los personajes, los niveles, entre otros aspectos, según \cite{}.
        \item Producción: Esta es la etapa en la cual se dessarrollan los assets de los videojuegos, estos assets incluyen: el arte visual del videojuego, la música, los efectos de sonido, la programación del motor del videojuego, entre otros aspectos, según \cite{}.
        \item Pruebas: El propósito de esta etapa es identificar y corregir errores en el videojuego, así como también mejorar la jugabilidad y la experiencia del usuario, según \cite{}.
        \item Beta: Aparte de las pruebas internas, en esta etapa se realizan pruebas externas con usuarios reales para obtener retroalimentación sobre el videojuego, según \cite{}.
        \item Release: En esta etapa se lanza el videojuego al público, ya sea a través de plataformas digitales o físicas, según \cite{}.
    \end{enumerate}

    \begin{figure}
        \centering
        \includegraphics[width=0.7\textwidth]{Imagenes/gdlc.png}
        \caption{Etapas del GDLC. Adaptado de \cite{}}
        \label{fig:gdlc_stages}
    \end{figure}
    
    Otra metodología es GAMED (Game Authoring Method for Educational Design) propuesta por \cite{}, la cual está enfocada en el desarrollo de videojuegos educativos y se basa en un enfoque iterativo e incremental. Además,
    
    \item ¿Qué mecánicas poseen los videojuegos educativos para poder enseñar temas educativos, especialmente los temas de ciencia y tecnología?
    \item ¿Qué evidencia y metodologías de resultados existen sobre la eficacia de los videojuegos educativos en la enseñanza de ciencias en contextos rurales o con recursos limitados?
    \item ¿Cómo los videojuegos educativos pueden integrar elementos culturales en su diseño?
    \item ¿Qué soluciones actuales se desarrollaron para la enseñanza de ciencias mediante el uso de videojuegos educativos?
\end{enumerate}

\section{Conclusiones}