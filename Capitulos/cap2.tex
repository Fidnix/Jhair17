\renewcommand{\baselinestretch}{1.5}
\onehalfspacing
\fancypagestyle{plain}{
\fancyhf{}
\rhead{\thepage}
\renewcommand{\headrulewidth}{0pt}
\renewcommand{\footrulewidth}{0pt}
}

\chapter{Marco Legal/Regulatorio/Conceptual/Otros}

\section{Introducción}

% Para el desarrollo y comprensión del contenido del presente proyecto, se realizó un marco conceptual, con el fin de conceptualizar los términos principales utilizados.

\section{Desarrollo del marco}

% \subsection{Videojuegos}

% \subsection{Videojuegos educativos/serios}

% Según  Corti(2006, citado en \cite{susi_educative_games}) los videojuegos serios son herramientas que permiten aprovechar la atracción de los usuarios finales por los videojuegos de computadoras para desarollar nuevos conocimientos o habilidades

% \subsection{Mecánicas de aprendizaje}

\subsection{Teoría del aprendizaje constructivista}

Esta teoría propone al humano como un agente activo de aprendizaje, el cual construye sus ideas a partir de ideas, conocimientos y creencias previas.

\cite{} señala que el conocimiento solo está en la mente humana y que no siempre concuerda con la realidad; esto porque el humano trata de entender su ambiente creando nuevos modelos mentales a partir de sus percepciones según \cite{}.

Entre los beneficios de considerar la teoría constructuvista como centro principal de aprendizaje según \cite{} son:

\begin{enumerate}
    \item Los niños aprenden y se divierten más al estar activamente involucrados en su aprendizaje.
    \item El aprendizaje funciona mejor cuando se trabaja sobre el entendimiento y pensamiento que en la memorización.
    \item El aprendizaje constructivista es transferible.
    \item El constructivismo otorga a los estudiantes mayor control sobre su aprendizaje, ya que este se basa en sus propias preguntas y exploraciones. Al estimular la creatividad, los estudiantes expresan el conocimiento de diversas formas y aumentan la probabilidad de retenerlo y aplicarlo en la vida real.
    \item El constructivismo motiva a los estudiantes al situar el aprendizaje en contextos reales, fomentando que cuestionen y apliquen su curiosidad.
    \item El constructivismo fortalece las habilidades sociales y de comunicación al fomentar la colaboración y el intercambio de ideas, preparando a los estudiantes para cooperar y desenvolverse en situaciones reales.

\end{enumerate}